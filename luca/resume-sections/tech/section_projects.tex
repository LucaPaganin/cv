\cvsection{\ifenglish Projects and Technical Experience\else Abilità tecniche e Progetti\fi}

%----------------------------------------------------------------------------------------
%	SECTION CONTENT
%----------------------------------------------------------------------------------------

\begin{cventries}

%------------------------------------------------

\cventry
{Developer} % Role
{SEYFERT: the SurvEY FishEr foRecast Tool} % Project
{Python, NumPy, SciPy, Pandas, Matplotlib} % Technologies
{2023 -- Present} % Date(s)
{ % Description(s)
\begin{cvitems}
  \ifenglish
  \item {Open-source Python tool for Fisher matrix forecasting of cosmological surveys.}
  \item {Cosmological analysis and predictions using advanced statistical methods and optimized numerical integration.}
  \else
  \item {Software Python open source per fare previsioni basate su matrici di Fisher sulle prestazioni di survey cosmologici.}
  \item {Analisi e predizioni cosmologiche basate su metodi statistici avanzati e tecniche ottimizzate di integrazione numerica.}
  \fi
  \item {Repository: \url{https://github.com/LucaPaganin/SEYFERT}}
\end{cvitems}
}

%------------------------------------------------

\cventry
{Developer} % Role
{Solar System Simulator} % Project
{C++, Python, Dash} % Technologies
{2022 -- Present} % Date(s)
{ % Description(s)
\begin{cvitems}
    \ifenglish
    \item {Simulator of solar system dynamics with 3D animation of planetary orbits.}
    \item {Numerical integration in C++ and web interface developed with Dash.}
    \else
    \item {Progetto che ho sviluppato per gli studenti delle scuole superiori allo stage di orientamento universitario.}
    \item {Simulatore della dinamica del sistema solare con animazione 3D delle orbite planetarie.}
    \item {Integrazione numerica in C++ e interfaccia web con Dash.}
    \fi
    \item {Repository: \url{https://github.com/LucaPaganin/SolarSystem/}}  
\end{cvitems}
}

%------------------------------------------------

\cventry
{Developer} % Role
{WebScraping Suite} % Project
{Python, HTML, JavaScript} % Technologies
{2021 -- Present} % Date(s)
{ % Description(s)
\begin{cvitems}
  \ifenglish
  \item {Collection of web scraping and data processing tools, including a Chrome extension for property listings and API scrapers for immobiliare.it.}
  \item {Supports data storage in CSV, SQLite, Azure Cosmos DB, and JSON formats.}
  \else
  \item {Collezione di strumenti per web scraping e data processing, inclusa estensione Chrome per listing immobiliari e scraper API per immobiliare.it.}
  \item {Supporto all'archiviazione su CSV, SQLite, Azure Cosmos DB e JSON.}
  \fi
  \item {Repository: \url{https://github.com/LucaPaganin/webscraping}}  
\end{cvitems}
}

%------------------------------------------------

\cventry
{Developer} % Role
{Personal Website} % Project
{Next.js, TypeScript, Tailwind CSS, DaisyUI, Framer Motion} % Technologies
{2023 -- Present} % Date(s)
{ % Description(s)
\begin{cvitems}
  \ifenglish
  \item {Multilingual personal website with dark mode, responsive design, and interactive components (timeline, carousel).}
  \else
  \item {Sito personale multilanguage con dark mode, design responsive e componenti interattivi (timeline, carousel).}
  \fi
  \item {Repository: \url{https://github.com/LucaPaganin/personal-website}}  
\end{cvitems}
}

%------------------------------------------------

\cventry
{Developer} % Role
{Collectify} % Project
{Flask, SQLAlchemy, SQLite, React} % Technologies
{2023 -- Present} % Date(s)
{ % Description(s)
\begin{cvitems}
  \ifenglish
  \item {Full stack web application for aggregating and presenting user data about collectibles inventory}
  \item {Backend written in Flask with SQLite database}
  \item {Frontend written in React}
  \else
  \item {Applicazione web basata su Flask per l'aggregazione e presentazione di dati utente tramite template dinamici e asset statici.}
  \item {Architettura MVC con routing in app.py e gestione di layout e statici in templates/.}
  \fi
  \item {Repository: \url{https://github.com/LucaPaganin/collectify}}  
\end{cvitems}
}

\end{cventries}