% Font configuration for better Italian accent support
% This file should be imported in resume_cv.tex after \documentclass but before beginning the document

% Define font feature settings for better accent support
\defaultfontfeatures{
  Ligatures=TeX,
  Path=\@fontdir,
  Scale=1.0,
  Extension=.ttf,
}

% Define Source Sans Pro with explicit features
\newfontfamily\sourcesansprofont[
  Path=\@fontdir,
  UprightFont=*-Regular.otf,
  ItalicFont=*-It.otf,
  BoldFont=*-Bold.otf,
  BoldItalicFont=*-BoldIt.otf,
  Scale=1.0,
  Ligatures=TeX,
  BoldItalicFeatures={SmallCapsFont=*-BoldIt.otf},
]{SourceSansPro}

% Define Source Sans Pro Light with explicit features
\newfontfamily\sourcesansprolightfont[
  Path=\@fontdir,
  UprightFont=*-Light.otf,
  ItalicFont=*-LightIt.otf,
  BoldFont=*-Semibold.otf,
  BoldItalicFont=*-SemiboldIt.otf,
  Scale=1.0,
  Ligatures=TeX,
  BoldItalicFeatures={SmallCapsFont=*-SemiboldIt.otf},
]{SourceSansPro}

% Override the default font commands with our new definitions
\renewcommand*{\bodyfont}{\sourcesansprofont}
\renewcommand*{\bodyfontlight}{\sourcesansprolightfont}
\renewcommand*{\footerfont}{\sourcesansprofont}

% Additional language-specific settings for polyglossia
\directlua{
  % Enable full Unicode character support
  fonts.handlers.otf.addfeature{
    name = "italics",
    type = "substitution",
    data = {
      ["à"] = "à",
      ["è"] = "è",
      ["é"] = "é",
      ["ì"] = "ì",
      ["ò"] = "ò",
      ["ù"] = "ù",
    }
  }
}

\ifenglish
  % English font settings
  \setmainfont[
    Path=\@fontdir,
    UprightFont=SourceSansPro-Regular.otf,
    BoldFont=SourceSansPro-Bold.otf,
    ItalicFont=SourceSansPro-It.otf,
    BoldItalicFont=SourceSansPro-BoldIt.otf,
    Mapping=tex-text,
    Ligatures=TeX
  ]{SourceSansPro}
\else
  % Italian font settings with accent support
  \setmainfont[
    Path=\@fontdir,
    UprightFont=SourceSansPro-Regular.otf,
    BoldFont=SourceSansPro-Bold.otf,
    ItalicFont=SourceSansPro-It.otf,
    BoldItalicFont=SourceSansPro-BoldIt.otf,
    Mapping=tex-text,
    Ligatures=TeX,
    Language=Italian,
    CharacterVariant={1,3},  % Enable character variants that might help with accents
    Contextuals=Alternate    % Enable contextual alternates
  ]{SourceSansPro}
\fi
